\chapter{related work}
\label{chp:2_literature}

In this chapter, related studies are given in detail.
Spatial transformations as a method for generating adversarial examples was first proposed in ~\cite{xiao2018spatially}, where it is shown that small displacements applied to input pixels can successfully fool a target network. However, using this method, even small displacements could cause visible distortions when the adjacent pixels are drifted towards different directions. As a remedy to this problem, use of  Total Variation~(TV) regularization~\cite{estrela2016total} was proposed. Application of TV regularization to the flow field pushes the neighboring displacement vectors to the same direction and, hence, produces smoother output. Similarly, Jordan et al. combined spatial transformations with \(l_\infty\) bounded attacks to forge stronger attacks with better perceptual quality. Croce et al. argued adding noise to smooth areas of an image causes visible artifacts and proposed "hiding" the perturbations at the locations with high spatial variations such as edges and corners~\cite{croce2019sparse}. As seen from Figure \ref{fig:diff}, perturbations made with our method naturally occurs in the places with high variations since it is based on local spatial transforms.

Utilizing perceptual colorspaces and metrics for imperceptible adversarial example generation is investigated in several studies. Aksoy et al. investigated additive noise based attacks on chrominance channels in YUV colorspace \cite{aksoy2019attack}, which is the analog counterpart of \(YC_{b}C_{r}\) space. Despite Pestana et al. found that adversarial perturbations are more highlighted in luminance channels in terms of the magnitude~\cite{Pestana2020-hm}, Aksoy et al. found that even suppressing the luminance perturbation, additive noise based attack on chrominance channels still successfully fool target networks, yet causes visible distortion. In our earlier work, we also explored spatial transformations to UV channels of YUV to generate imperceptible adversarial examples~\cite{aydin2019imperceptible} and we extend this work by exploring \(YC_{b}C_{r}\) space as well as perceptually uniform CIELAB space and measuring structured similarity metrics such as SSIM~\cite{wang2004image} and MS-SSIM~\cite{wang2003multiscale} between benign images and adversarially generated images. Karli et al. leveraged perceptual metric LPIPS~\cite{zhang2018unreasonable} to improve the quality of adversarial examples. Since LPIPS is a differentiable metric, they used gradient based optimization to minimize LPIPS alongside the adversarial loss. Similarly, Zhao et al. replaced CIEDE2000 perceptual distance metric~\cite{luo2001development} with \(\mathcal{L}_{p}\) norm constraint in Carlini \& Wagner attack to produce perceptually close adversarial examples.

Unlike these methods, the attack proposed in this paper does not rely on auxiliary losses or explicit perceptual distance terms in optimization process to produce examples with high perceptual quality. In addition, it does not require regularization, unlike spatial transformation based methods such as ~\cite{xiao2018spatially}, due to its intrinsic imperceptibility. It should be noted that the existing spatial transformation based methods, as well as our work, does not utilize limited degree of freedom transformations such as rotation, translation or scaling that can be formulated as a \(4\times4\) transformation matrix~\cite{jaderberg2015spatial}. In that formulation, the flow field \(f \in \mathbb{R}^{2\times H \times W}\) is calculated using the transformation matrix. Instead, we directly define and optimize flow field, where the number of parameters is equal to twice number of pixels in the input image since there is an x and y component for each pixel. Application and optimization of flow field is explained in the Section \ref{section:methodology}.

\section{Related Work Section I}
\label{Section2.1}


